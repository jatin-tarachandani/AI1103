\documentclass[twocolumn]{article}

\usepackage{setspace}
\usepackage{gensymb}
\singlespacing
\usepackage[cmex10]{amsmath}

\usepackage{amsthm}
\usepackage{amsmath}
\usepackage{amssymb}
\usepackage{mathrsfs}
\usepackage{txfonts}
\usepackage{stfloats}
\usepackage{bm}
\usepackage{cite}
\usepackage{cases}
\usepackage{subfig}

\usepackage{longtable}
\usepackage{multirow}

\usepackage{enumitem}
\usepackage{mathtools}
\usepackage{steinmetz}
\usepackage{tikz}
\usepackage{circuitikz}
\usepackage{verbatim}
\usepackage{tfrupee}
\usepackage[breaklinks=true]{hyperref}
\usepackage{graphicx}
\usepackage{tkz-euclide}

\usetikzlibrary{calc,math}
\usepackage{listings}
    \usepackage{color}                                            %%
    \usepackage{array}                                            %%
    \usepackage{longtable}                                        %%
    \usepackage{calc}                                             %%
    \usepackage{multirow}                                         %%
    \usepackage{hhline}                                           %%
    \usepackage{ifthen}                                           %%
    \usepackage{lscape}     
\usepackage{multicol}
\usepackage{chngcntr}

\DeclareMathOperator*{\Res}{Res}

\renewcommand\thesection{\arabic{section}}
\renewcommand\thesubsection{\thesection.\arabic{subsection}}
\renewcommand\thesubsubsection{\thesubsection.\arabic{subsubsection}}

\renewcommand\thesectiondis{\arabic{section}}
\renewcommand\thesubsectiondis{\thesectiondis.\arabic{subsection}}
\renewcommand\thesubsubsectiondis{\thesubsectiondis.\arabic{subsubsection}}


\hyphenation{op-tical net-works semi-conduc-tor}
\def\inputGnumericTable{}                                 %%

\lstset{
%language=C,
frame=single, 
breaklines=true,
columns=fullflexible
}

\newcommand{\BEQA}{\begin{eqnarray}}
\newcommand{\EEQA}{\end{eqnarray}}
\newcommand{\define}{\stackrel{\triangle}{=}}
\bibliographystyle{IEEEtran}
\raggedbottom
\setlength{\parindent}{0pt}
\providecommand{\mbf}{\mathbf}
\providecommand{\pr}[1]{\ensuremath{\Pr\left(#1\right)}}
\providecommand{\qfunc}[1]{\ensuremath{Q\left(#1\right)}}
\providecommand{\sbrak}[1]{\ensuremath{{}\left[#1\right]}}
\providecommand{\lsbrak}[1]{\ensuremath{{}\left[#1\right.}}
\providecommand{\rsbrak}[1]{\ensuremath{{}\left.#1\right]}}
\providecommand{\brak}[1]{\ensuremath{\left(#1\right)}}
\providecommand{\lbrak}[1]{\ensuremath{\left(#1\right.}}
\providecommand{\rbrak}[1]{\ensuremath{\left.#1\right)}}
\providecommand{\cbrak}[1]{\ensuremath{\left\{#1\right\}}}
\providecommand{\lcbrak}[1]{\ensuremath{\left\{#1\right.}}
\providecommand{\rcbrak}[1]{\ensuremath{\left.#1\right\}}}
\theoremstyle{remark}
\newtheorem{rem}{Remark}
\newcommand{\sgn}{\mathop{\mathrm{sgn}}}
\providecommand{\abs}[1]{\vert#1\vert}
\providecommand{\res}[1]{\Res\displaylimits_{#1}} 
\providecommand{\norm}[1]{\lVert#1\rVert}
%\providecommand{\norm}[1]{\lVert#1\rVert}
\providecommand{\mtx}[1]{\mathbf{#1}}
\providecommand{\mean}[1]{E[ #1 ]}
\providecommand{\fourier}{\overset{\mathcal{F}}{ \rightleftharpoons}}
%\providecommand{\hilbert}{\overset{\mathcal{H}}{ \rightleftharpoons}}
\providecommand{\system}{\overset{\mathcal{H}}{ \longleftrightarrow}}
	%\newcommand{\solution}[2]{\textbf{Solution:}{#1}}
\newcommand{\solution}{\noindent \textbf{Solution: }}
\newcommand{\cosec}{\,\text{cosec}\,}
\providecommand{\dec}[2]{\ensuremath{\overset{#1}{\underset{#2}{\gtrless}}}}
\newcommand{\myvec}[1]{\ensuremath{\begin{pmatrix}#1\end{pmatrix}}}
\newcommand{\mydet}[1]{\ensuremath{\begin{vmatrix}#1\end{vmatrix}}}
\numberwithin{equation}{subsection}
\makeatletter
\@addtoreset{figure}{problem}
\makeatother
\let\StandardTheFigure\thefigure
\let\vec\mathbf
\renewcommand{\thefigure}{\theproblem}
\def\putbox#1#2#3{\makebox[0in][l]{\makebox[#1][l]{}\raisebox{\baselineskip}[0in][0in]{\raisebox{#2}[0in][0in]{#3}}}}
     \def\rightbox#1{\makebox[0in][r]{#1}}
     \def\centbox#1{\makebox[0in]{#1}}
     \def\topbox#1{\raisebox{-\baselineskip}[0in][0in]{#1}}
     \def\midbox#1{\raisebox{-0.5\baselineskip}[0in][0in]{#1}}

\title{Assignment 4}
\author{Jatin Tarachandani-CS20BTECH11021}
\date{March 2021}
\begin{document}
\maketitle
Download latex codes from 
%
\begin{lstlisting}
https://github.com/jatin-tarachandani/AI1103/blob/main/Assignment_4/main.tex
\end{lstlisting}
\section{Problem Statement}
CSIR UGC NET Dec 2012 Q60

Men arrive in a queue according to a Poisson process with rate $\lambda_1$ and women arrive in the same queue according to another Poisson process with rate $\lambda_2$. The arrivals of men and women are independent. The probability that the first person to arrive in the queue is a man is:
\begin{enumerate}
\item \dfrac{\lambda_1}{\lambda_1+\lambda_2}

\item \dfrac{\lambda_2}{\lambda_1+\lambda_2}

\item \dfrac{\lambda_1}{\lambda_2}

\item \dfrac{\lambda_2}{\lambda_1} 
    
\end{enumerate}
\section{Solution}
Let $X$ and $Y$ be Poisson random variables, with the values X takes being the number of men joining the queue in an arbitrary time $t$, and the values Y takes being the number of women joining the queue in an arbitrary time $t$.
\begin{align}
    Pr\brak{X=i}&=\frac{\lambda_1^i\cdot e^{-\lambda_1}}{i!}\\
    Pr\brak{Y=i}&=\frac{\lambda_2^i\cdot e^{-\lambda_2}}{i!}
\end{align}
For 2 independent Poisson distributions with means $\lambda_1$ and $\lambda_2$, the simultaneous distribution can be represented by:
\begin{align}
    Pr\brak{X+Y=i}=\frac{(\lambda_1+\lambda_2)^i\cdot e^{-(\lambda_1+\lambda_2})}{i!}
\end{align}
Now we take conditional probability that if only one person entered the queue within a certain time t, then the probability of them being a man and not a woman is given by:
\begin{align}
    Pr\brak{X=1|(X+Y)=1}&=\frac{Pr\brak{(X=1) + (Y=0)}}{Pr\brak{X+Y=1}}\\
\end{align}
Since X and Y are independent, 
\begin{align}
    Pr\brak{X=1|(X+Y)=1}&=\frac{Pr\brak{X=1}\cdot Pr\brak{Y=0}}{Pr\brak{X+Y=1}}\\
    &=\frac{\frac{\lambda_1^1\cdot e^{-\lambda_1}}{1!}\cdot \frac{\lambda_2^0\cdot e^{-\lambda_2}}{0!}}{\frac{(\lambda_1+\lambda_2)^1\cdot e^{-(\lambda_1+\lambda_2})}{1!}}\\
    &=\frac{lambda_1}{\lambda_1+\lambda_2}
\end{align}
 The probability that the first person to arrive in the queue is a man is option A, i.e $\frac{lambda_1}{\lambda_1+\lambda_2}$

\end{document}

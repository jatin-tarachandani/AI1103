\documentclass[twocolumn]{article}

\usepackage[utf8]{inputenc}
\usepackage{amsmath}
\usepackage{mathtools}
\usepackage{listings}
    \usepackage{color}                                            %%
    \usepackage{array}                                            %%
    \usepackage{longtable}                                        %%
    \usepackage{calc}                                             %%
    \usepackage{multirow}                                         %%
    \usepackage{hhline}                                           %%
    \usepackage{ifthen}                                           %%
    \usepackage{lscape}     
\usepackage{multicol}
\usepackage{chngcntr}
\lstset{
%language=C,
frame=single, 
breaklines=true,
columns=fullflexible
}

\title{Assignment 2 - GATE problem 49}
\author{Jatin Tarachandani-CS20BTECH11021}
\date{March 2021}
\begin{document}
\maketitle
Download the python codes from 
\begin{lstlisting}
https://github.com/jatin-tarachandani/AI1103/blob/main/Assignment%202/codes/Assignment_2.py
\end{lstlisting}
%
and latex codes from 
%
\begin{lstlisting}
https://github.com/jatin-tarachandani/AI1103/blob/main/Assignment%202/main.tex
\end{lstlisting}
\section{Problem Statement}
A fair coin is tossed 3 times in succession. If the first toss is a head, then the probability of getting exactly two heads in three tosses is?
\section{Solution}
We can see that if the first toss is guaranteed to be a head, then the problem is reduced to finding the probability of getting one head in 2 coin tosses, since all the 3 trials are independent.

Let $K=\{0, 1, 2\}$ be the random variable denoting the number of heads obtained in 2 tosses of a fair coin. The event consists of multiple Bernoulli trials, therefore it can be represented by a binomial distribution b(n,p).
In b(n,p), Pr($K=i)= \binom{n}{i}p^i \cdot (1-p)^{n-i}$. Here $n=2$, $p=0.5$.

\begin{align}
    Pr(K=0)&=\binom{2}{0} 0.5^2=\frac{1}{4}\\
    Pr(K=1)&=\binom{2}{1} 0.5^2=\frac{1}{2}\\
    Pr(K=2)&=\binom{2}{2} 0.5^2=\frac{1}{4}
\end{align}
 We can see that the probability of getting 1 head in 2 tosses is 
 \begin{align*}
 Pr(K=1)&=\binom{2}{1} 0.5^2
 &=\frac{1}{2}\\
 \end{align*}
 
The probability of getting exactly 2 heads in 3 tosses, if the first toss is a head, is 0.5.
\end{document}
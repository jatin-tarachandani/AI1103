\documentclass[twocolumn]{article}

\usepackage[utf8]{inputenc}
\usepackage{amsmath}
\usepackage{mathtools}
\usepackage{listings}
    \usepackage{color}                                            %%
    \usepackage{array}                                            %%
    \usepackage{longtable}                                        %%
    \usepackage{calc}                                             %%
    \usepackage{multirow}                                         %%
    \usepackage{hhline}                                           %%
    \usepackage{ifthen}                                           %%
    \usepackage{lscape}     
\usepackage{multicol}
\usepackage{chngcntr}
\lstset{
%language=C,
frame=single, 
breaklines=true,
columns=fullflexible
}

\title{Assignment 2 - GATE problem 49}
\author{Jatin Tarachandani-CS20BTECH11021}
\date{March 2021}
\begin{document}
\maketitle
Download the python codes from 
\begin{lstlisting}
https://github.com/jatin-tarachandani/AI1103/blob/main/Assignment%202/codes/Assignment_2.py
\end{lstlisting}
%
and latex codes from 
%
\begin{lstlisting}
https://github.com/jatin-tarachandani/AI1103/blob/main/Assignment%202/main.tex
\end{lstlisting}
\section{Problem Statement}
A fair coin is tossed 3 times in succession. If the first toss is a head, then the probability of getting exactly two heads in three tosses is?
\section{Solution}
Let $K \in \{0, 1, 2, 3\}$ be the random variable denoting the possible numbers of heads we obtain in three consecutive tosses of the coin. Let $L \in \{0, 1\}$ be the random variable denoting the result of the first flip, with 0 representing a result of tails.
We see that K represents the probabilities of getting a certain number of successes in 3 distinct Bernoulli trials, so we can find the probabilities for K to take its possible values via a binomial distribution b(n, p).
We know that for a binomial distribution, $Pr(K=r)=\binom{n}{r}p^r (1-p)^{n-r}$. Here, $n=3, p=0.5$.
\begin{align}
    Pr(K=0)&=\binom{3}{0}0.5^0 0.5^{3-0}=0.125\\
    Pr(K=1)&=\binom{3}{1}0.5^1 0.5^{3-1}=0.375\\
    Pr(K=2)&=\binom{3}{2}0.5^2 0.5^{3-2}=0.375\\
    Pr(K=3)&=\binom{3}{3}0.5^3 0.5^{3-3}=0.125\\
    Pr(L=1|K=0)&=0\\
    Pr(L=1|K=1)&=\frac{1}{3} \equiv 0.33\\
    Pr(L=1|K=2)&=\frac{2}{3} \equiv 0.67\\
    Pr(L=1|K=3)&=1
\end{align}
Using Bayes theorem, we get:
\begin{align*}
    Pr(K=2|L=1)&=\frac{Pr(L=1|K=2)\cdot Pr(K=2)}{\sum_{i=0}^{3} Pr(L=1|K=i)\cdot Pr(K=i)}\\
    &=\frac{\frac{2}{3}\cdot \binom{3}{2}p^2 (1-p)^{3-2}}{\splitfrac{0\cdot\binom{3}{0}p^0 (1-p)^{3}+\frac{1}{3}\cdot\binom{3}{1}p^1 (1-p)^{2}}{+\frac{2}{3}\cdot \binom{3}{2}p^2 (1-p)^{1}+1\cdot \binom{3}{3}p^3 (1-p)^{0}}}
\end{align*}
Since $p=0.5=1-p$, we can replace $1-p$ by $p$.
\begin{align*}
     Pr(K=2|L=1)&=\frac{0.67\cdot 3\cdot  p^3}{\splitfrac{0\cdot p^3+0.33\cdot3\cdot p^3}{+0.67\cdot 3 \cdot p^3+1\cdot p^3}}\\
     &=\frac{2 \cdot p^3}{4 \cdot p^3}\\
    &=0.5.
\end{align*}

The probability of getting exactly 2 heads in 3 tosses, if the first toss is a head, is 0.5.
\end{document}

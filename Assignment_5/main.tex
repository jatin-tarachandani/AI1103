\documentclass[twocolumn]{article}

\usepackage{setspace}
\usepackage{gensymb}
\singlespacing
\usepackage[cmex10]{amsmath}

\usepackage{amsthm}
\usepackage{amsmath}
\usepackage{amssymb}
\usepackage{mathrsfs}
\usepackage{txfonts}
\usepackage{stfloats}
\usepackage{bm}
\usepackage{cite}
\usepackage{cases}
\usepackage{subfig}

\usepackage{longtable}
\usepackage{multirow}

\usepackage{enumitem}
\usepackage{mathtools}
\usepackage{steinmetz}
\usepackage{tikz}
\usepackage{circuitikz}
\usepackage{verbatim}
\usepackage{tfrupee}
\usepackage[breaklinks=true]{hyperref}
\usepackage{graphicx}
\usepackage{tkz-euclide}

\usetikzlibrary{calc,math}
\usepackage{listings}
    \usepackage{color}                                            %%
    \usepackage{array}                                            %%
    \usepackage{longtable}                                        %%
    \usepackage{calc}                                             %%
    \usepackage{multirow}                                         %%
    \usepackage{hhline}                                           %%
    \usepackage{ifthen}                                           %%
    \usepackage{lscape}     
\usepackage{multicol}
\usepackage{chngcntr}

\DeclareMathOperator*{\Res}{Res}

\renewcommand\thesection{\arabic{section}}
\renewcommand\thesubsection{\thesection.\arabic{subsection}}
\renewcommand\thesubsubsection{\thesubsection.\arabic{subsubsection}}

\renewcommand\thesectiondis{\arabic{section}}
\renewcommand\thesubsectiondis{\thesectiondis.\arabic{subsection}}
\renewcommand\thesubsubsectiondis{\thesubsectiondis.\arabic{subsubsection}}


\hyphenation{op-tical net-works semi-conduc-tor}
\def\inputGnumericTable{}                                 %%

\lstset{
%language=C,
frame=single, 
breaklines=true,
columns=fullflexible
}

\newcommand{\BEQA}{\begin{eqnarray}}
\newcommand{\EEQA}{\end{eqnarray}}
\newcommand{\define}{\stackrel{\triangle}{=}}
\newcommand{\comb}[2]{{}^{#1}\mathrm{C}_{#2}}
\bibliographystyle{IEEEtran}
\raggedbottom
\setlength{\parindent}{0pt}
\providecommand{\mbf}{\mathbf}
\providecommand{\pr}[1]{\ensuremath{\Pr\left(#1\right)}}
\providecommand{\qfunc}[1]{\ensuremath{Q\left(#1\right)}}
\providecommand{\sbrak}[1]{\ensuremath{{}\left[#1\right]}}
\providecommand{\lsbrak}[1]{\ensuremath{{}\left[#1\right.}}
\providecommand{\rsbrak}[1]{\ensuremath{{}\left.#1\right]}}
\providecommand{\brak}[1]{\ensuremath{\left(#1\right)}}
\providecommand{\lbrak}[1]{\ensuremath{\left(#1\right.}}
\providecommand{\rbrak}[1]{\ensuremath{\left.#1\right)}}
\providecommand{\cbrak}[1]{\ensuremath{\left\{#1\right\}}}
\providecommand{\lcbrak}[1]{\ensuremath{\left\{#1\right.}}
\providecommand{\rcbrak}[1]{\ensuremath{\left.#1\right\}}}
\theoremstyle{remark}
\newtheorem{rem}{Remark}
\newtheorem{theorem}{Theorem}[section]
\newtheorem{corollary}{Corollary}[theorem]
\newtheorem{lemma}[theorem]{Lemma}

\theoremstyle{definition}
\newtheorem{definition}{Definition}[section]

\newcommand{\sgn}{\mathop{\mathrm{sgn}}}
\providecommand{\abs}[1]{\vert#1\vert}
\providecommand{\res}[1]{\Res\displaylimits_{#1}} 
\providecommand{\norm}[1]{\lVert#1\rVert}
%\providecommand{\norm}[1]{\lVert#1\rVert}
\providecommand{\mtx}[1]{\mathbf{#1}}
\providecommand{\mean}[1]{E[ #1 ]}
\providecommand{\fourier}{\overset{\mathcal{F}}{ \rightleftharpoons}}
%\providecommand{\hilbert}{\overset{\mathcal{H}}{ \rightleftharpoons}}
\providecommand{\system}{\overset{\mathcal{H}}{ \longleftrightarrow}}
	%\newcommand{\solution}[2]{\textbf{Solution:}{#1}}
\newcommand{\solution}{\noindent \textbf{Solution: }}
\newcommand{\cosec}{\,\text{cosec}\,}
\providecommand{\dec}[2]{\ensuremath{\overset{#1}{\underset{#2}{\gtrless}}}}
\newcommand{\myvec}[1]{\ensuremath{\begin{pmatrix}#1\end{pmatrix}}}
\newcommand{\mydet}[1]{\ensuremath{\begin{vmatrix}#1\end{vmatrix}}}
\numberwithin{equation}{subsection}
\makeatletter
\@addtoreset{figure}{problem}
\makeatother
\let\StandardTheFigure\thefigure
\let\vec\mathbf
\renewcommand{\thefigure}{\theproblem}
\def\putbox#1#2#3{\makebox[0in][l]{\makebox[#1][l]{}\raisebox{\baselineskip}[0in][0in]{\raisebox{#2}[0in][0in]{#3}}}}
     \def\rightbox#1{\makebox[0in][r]{#1}}
     \def\centbox#1{\makebox[0in]{#1}}
     \def\topbox#1{\raisebox{-\baselineskip}[0in][0in]{#1}}
     \def\midbox#1{\raisebox{-0.5\baselineskip}[0in][0in]{#1}}

\title{Assignment 5}
\author{Jatin Tarachandani-CS20BTECH11021}
\date{May 2021}
\begin{document}
\maketitle
Download latex codes from 
%
\begin{lstlisting}
https://github.com/jatin-tarachandani/AI1103/blob/main/Assignment_5/main.tex
\end{lstlisting}
\section{Problem Statement}
Let $X$, $Y$, have the joint discrete distribution such that $X|Y=y \sim$ Binomial($y$, 0.5) and $Y\sim$ Poisson($\lambda$), $\lambda>0$, where $\lambda$ is an unknown parameter. Let $T=T(X, Y)$ be any unbiased estimator of $\lambda$. Then
\begin{enumerate}
    \item  $Var(T) \leq Var(Y)  \text{ for all } \lambda$
    \item $Var(T) \geq Var(Y) \text{ for all } \lambda$
    \item $Var(T) \geq \lambda \text{ for all } \lambda$
    \item $Var(T) = Var(Y) \text{ for all } \lambda$
\end{enumerate}

\section{Solution}
\begin{definition}
Suppose we have an estimator $T(X, Y)$, acting on a set of 2 RVs $X, Y$; which estimates a parameter $\theta$, then if
\begin{equation}
    E\brak{T(X, Y)}=\theta
\end{equation}

the estimator is unbiased. 
\end{definition}

\begin{lemma}
Since Y has a Poisson distribution, we know that:
\begin{align}\label{Var=lam}
Var(Y)=\lambda
\end{align}
\end{lemma}

From Bayes Theorem,
\begin{align}
    \pr{X=x,Y=y}&=\pr{Y=y}  \pr{X=x|Y=y}\\
    &= \comb{y}{x} \frac{1}{2^y} \frac{\lambda^y}{y!} e^{- \lambda}
\end{align}
Let us represent $ \pr{X=x,Y=y}$ as a function of $x, y$, and $\lambda$: $f(x, y;\lambda)$.

\begin{definition} \label{def CRB}
If $T(X, Y)$ is an unbiased estimator of a parameter $\lambda$, the Cramer-Rao bound states that:
\begin{align}
    Var\brak{T\brak{X,Y}}\geq -\frac{1}{E\brak{\frac{\partial^2 \ln(f(x,y;\lambda))}{\partial\lambda^2}}}
\end{align}
\end{definition}

From applying $\ref{def CRB}$ on $T(X,Y)$,
\begin{align}
    Var(T(X, Y))&\geq -\frac{1}{E\brak{\frac{\partial^2 \ln(f(x,y;\lambda))}{\partial\lambda^2}}}\\
    &\geq-\frac{1}{E\brak{-\frac{y}{\lambda^2}}}\\
    &\geq\frac{\lambda^2}{E\brak{y}}\\
    &\geq \lambda \label{CramerRao}
\end{align}
because the expectation value of a Poisson distribution with parameter $\lambda$ is $\lambda$.

The correct options are options (2) and (3), since by \eqref{CramerRao}, we see that the variance of $T$ is $\geq \lambda=Var(Y)$ (from \eqref{Var=lam}). (1) and (4) do not hold for all $\lambda$.



\end{document}

